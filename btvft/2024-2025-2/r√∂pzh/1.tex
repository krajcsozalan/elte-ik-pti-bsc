\documentclass{article}
\usepackage[utf8]{inputenc}
\usepackage[T1]{fontenc}
\usepackage[margin=1in]{geometry}
\usepackage[hungarian]{babel}
\usepackage{amsthm, amsmath, amssymb}
\theoremstyle{definition}
\newtheorem*{definition*}{Definíció}
\newtheorem*{theorem*}{Tétel}
\title{Elméleti kérdések az 1.\ röpZH-hoz}
\author{Többváltozós függvénytan (B szakirány)}
\date{2025.\ február 17-21.}
\begin{document}
\maketitle
\tableofcontents
\goodbreak
\section{Normált tér fogalma}
\begin{definition*}
	Legyen $X \neq \emptyset$ egy lineáris tér $\mathbb{R}$ felett és
	$\| . \|: X \to \mathbb{R}$ olyan függvény,
	amelyre minden $x, y \in X$ és $\lambda \in \mathbb{R}$ esetén igaz, hogy
	\begin{itemize}
		\item $\| x \| \geq 0 \quad$ és $\quad \| x \| = 0 \iff x = 0$
		\item $\| \lambda \cdot x \| = |\lambda| \cdot \| x \|$
		\item $\| x + y \| \leq \| x \| + \| y \|$
	\end{itemize}
	Ekkor az $(X, \| . \|)$ együttest normált térnek nevezzük.
\end{definition*}
\end{document}
